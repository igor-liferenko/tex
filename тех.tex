\catcode`\{=1
\catcode`\}=2
\catcode`\#=6
\def\transformfont#1#2#3#4#5 {%
  \if#1c%
    \if#3m%
      cmm#4#5 % cmmi[5-10], cmmib10
    \else
      \if#3b%
        \if#4s%
          cmbs#5 % cmbsy10
        \else
          omb#4#5
        \fi
      \else
        \if#3s%
          \if#4y%
            cmsy#5 % cmsy[5-10]
          \else
            oms#4#5
          \fi
        \else
          \if#3e%
            cme#4#5 % cmex10
          \else
            om#3#4#5
          \fi
        \fi
      \fi
    \fi
  \else
    #1#2#3#4#5 % manfnt
  \fi}
\let\originalfont=\font
\def\font#1=#2 {\originalfont#1=\transformfont#2 }

\input plain

\let\font=\originalfont
\let\originalfont=\undefined
\let\transformfont=\undefined

\def\\ #1 #2 {
  \lccode`#1=`#2
  \lccode`#2=`#2
  \uccode`#1=`#1
  \uccode`#2=`#1
  \sfcode`#1=999
}

\\ А а
\\ Б б
\\ В в
\\ Г г
\\ Д д
\\ Е е
\\ Ё ё
\\ Ж ж
\\ З з
\\ И и
\\ Й й
\\ К к
\\ Л л
\\ М м
\\ Н н
\\ О о
\\ П п
\\ Р р
\\ С с
\\ Т т
\\ У у
\\ Ф ф
\\ Х х
\\ Ц ц
\\ Ч ч
\\ Ш ш
\\ Щ щ
\\ Ъ ъ
\\ Ы ы
\\ Ь ь
\\ Э э
\\ Ю ю
\\ Я я

\righthyphenmin=2
\language1
\input hyph-ru

\catcode`^^ff=\active
\catcode`@=11
\def^^ff{\nobreak\kern\z@-\nobreak\kern\z@}
\catcode`@=12

\font\TENRM=OMR10
\font\SEVENRM=OMR7
\font\FIVERM=OMR5
\font\TENBF=OMBX10
\font\SEVENBF=OMBX7
\font\FIVEBF=OMBX5
\font\TENSL=OMSL10
\font\TENTT=OMTT10
\font\TENIT=OMTI10
