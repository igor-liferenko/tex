  \catcode`\{=1
  \catcode`\}=2
  \catcode`\#=6
  \def\transformfont#1#2#3#4#5 {%
    \if#1c%
      \if#2m%
        \if#3m%
          \if#4i%
            cmmi#5
          \else
            lhm#4#5
          \fi
        \else
          \if#3b%
            \if#4s%
              cmbs#5
            \else
              lhb#4#5
            \fi
          \else
            \if#3s%
              \if#4y%
                cmsy#5
              \else
                lhs#4#5
              \fi
            \else
              \if#3e%
                \if#4x%
                  cmex#5
                \else
                  lhe#4#5
                \fi
              \else
                \if#3t%
                  \if#4e%
                    cmte#5
                  \else
                    lht#4#5
                  \fi
                \else
                  lh#3#4#5
                \fi
              \fi
            \fi
          \fi
        \fi
      \else
        c#2#3#4#5
      \fi
    \else
      #1#2#3#4#5
    \fi}
  \let\originalfont\font
  \def\font#1=#2 {\originalfont#1=\transformfont#2 }

\input plain

\newfam\cyrfam % to use one-letter identifiers in CWEB

  \let\font\originalfont
  \let\originalfont\undefined
  \let\transformfont\undefined

  \def\set #1 #2 {
    \lccode`#1=`#2
    \lccode`#2=`#2
    \uccode`#1=`#1
    \uccode`#2=`#1
    \sfcode`#1=999
    \sfcode`#2=1000

    \count255=256
    \multiply\count255 by \cyrfam
    \advance\count255 by `#1
    \mathcode`#1=\count255
    \count255=256
    \multiply\count255 by \cyrfam
    \advance\count255 by `#2
    \mathcode`#2=\count255
  }

  \textfont\cyrfam=\tenit % there are no cyrillic math fonts, so use ordinary

  \set А а

  \set Б б

  \set В в

  \set Г г

  \set Д д

  \set Е е

  \set Ё ё

  \set Ж ж

  \set З з

  \set И и

  \set Й й

  \set К к

  \set Л л

  \set М м

  \set Н н

  \set О о

  \set П п

  \set Р р

  \set С с

  \set Т т

  \set У у

  \set Ф ф

  \set Х х

  \set Ц ц

  \set Ч ч

  \set Ш ш

  \set Щ щ

  \set Ъ ъ

  \set Ы ы

  \set Ь ь

  \set Э э

  \set Ю ю

  \set Я я

  \let\set=\undefined

  % Hyphenation:

  \righthyphenmin=2
  \language1
  \input ruhyphal
