\input tugboat.sty

\title * {\ninebf UTF}-8 installations of \TeX *
\author * Igor Liferenko *
\address * \tt igor.liferenko (at) gmail dot com *

\article

\head * Abstract *

\indent In its design \TeX\ has the concepts of ``internal
encoding'' and ``external encoding''. This fact
allows \TeX\ to work with any encoding.

We use Unicode as \TeX's external encoding.
Then we change the necessary parts of \TeX\ to
use \hbox{\UTF-8} as the
input/output encoding.

The resulting implementation passes
the |trip| test.

\head * 1. Initialization *

Note: we use the {\sl web2w\/} [1] implementation
of \TeX, but the ideas described here can be applied
to any implementation.

First, we change the data type of characters in text files to
|wchar_t| to accommodate Unicode values.

Background: this predefined C type allocates four bytes
per character (on most systems). Character constants of this type are
written as |L'...'| and string constants as |L"..."|.

(For brevity, in the diffs following, the original code from
{\sl web2w\/}'s |ctex.w| source is preceded with |<| characters, and the
new code with |>|. Both are sometimes reformatted for presentation
in this article, and for readability we sometimes leave a blank line
between the pieces. The actual implementation uses the file
|utex.patch| [2].)
\verbatim
< @d text_char unsigned char
> @d text_char wchar_t
\endverbatim
\medskip
Use values from the
basic multilingual plane (\acro{BMP}) of Unicode.
\verbatim
< @d last_text_char 255
> @d last_text_char 65535
\endverbatim
\medskip
Then we change the size of the {\it xord\/} array [3] to
$2^{16}$ bytes.
\verbatim
< ASCII_code xord[256];
> ASCII_code xord[65536];
\endverbatim
\medskip
Elements in the {\it xchr\/} array [3]
are overridden using
the file |mapping.w|.
||@i mapping.w||
This file specifies the character(s)
required for a particular
installation of \TeX, for example:
\smallskip
\indent{\font\tt=cmtt9 \tt xchr[0xf1] = L'\"e';}
\smallskip
\noindent A complete example of |mapping.w| is here:
||https://github.com/igor-liferenko/cweb||
\medskip
Let's make |char| and |text_char| the same type (by analogy with \S19 in [3]).
\verbatim
< char TEX_format_default
<    [1+format_default_length+1]
<  =" TeXformats/plain.fmt";

> wchar_t TEX_format_default
>    [1+format_default_length+1]
>  =L" TeXformats/plain.fmt";

< char months[]=" JANFEBMARAPRMAYJUNJUL...
> wchar_t months[]=L" JANFEBMARAPRMAYJU... 
\endverbatim
\medskip
It remains to set the |LC_CTYPE| locale category, on which
depends the behavior of the C~library functions used below
||setlocale(LC_CTYPE, "C.UTF-8");||
and to add the necessary headers.
\verbatim
#include <wchar.h>
#include <locale.h>
\endverbatim

\head * 2. Input *

For automatic conversion from \UTF-8 to Unicode,
text files (including the terminal) must be
read with the C~library function \hbox{\it fgetwc\/} [4].

In |ctex.w| the macro {\it get\/}
is used for reading text files, as well as
font metric and format files.

Text files are read in the functions
{\it a\_open\_in\/} and
{\it input\_ln\/}.
In
{\it a\_open\_in\/} we replace the macro
{\it reset\/}
with its expansion and then in both functions
we change |get((*f))| to |(*f).d=fgetwc((*f).f)|

\head * 3. Output *

In the places where values of the type |wchar_t| are printed, change
|%c| to |%lc| [4].
\smallskip
\verbatim
< wterm("%c",xchr[s]);
> wterm("%lc",xchr[s]);

< wlog("%c",xchr[s]);
> wlog("%lc",xchr[s]);

< write(write_file[selector],"%c",xchr[s]);
> write(write_file[selector],"%lc",xchr[s]);

< wlog("%c",months[k]);
> wlog("%lc",months[k]);
\endverbatim

\head * 4. The file name buffer *

The name of the file to be opened, which is
stored in the {\it name\_of\_file\/} buffer,
must be encoded in \hbox{\UTF-8}.
Therefore, each character passed to {\it append\_to\_name\/},
before being added to {\it name\_of\_file\/},
must
be converted to \hbox{\UTF-8}. This is done using
the C~library function {\it wctomb\/} [4].

In the {\it append\_to\_name\/} macro, the variable
{\it k\/}
is used as the index into the {\it name\_of\_file\/} buffer
where the last byte was stored. Originally, {\it k\/} was
always increased and provisions were made
that characters will not be written beyond the end of buffer
(which has the index {\it file\_name\_size\/});
{\it name\_length\/} was then set to the minimal value
between {\it k\/} and {\it file\_name\_size\/}.

We cannot do the same in our implementation, because we may reach
the end of the buffer
in the midst of
a multibyte character. Instead, if the next multibyte
character does not fit into the buffer, we prevent {\it k\/} from
being increased by negating its value:
\verbatim
< @d append_to_name(X) { c=X;incr(k);
<   if (k <= file_name_size)
<     name_of_file[k]=xchr[c]; } 

> @d append_to_name(X) { c=X;
>   if (k >= 0) { /* try to append? */
>     char mb[MB_CUR_MAX];
>     int len = wctomb(mb, xchr[c]);
>     if (k+len <= file_name_size)
>       for (int i = 0; i < len; i++)
>         name_of_file[++k] = mb[i];
>     else
>       k = -k; /* freeze k */ } }
\endverbatim
\medskip
In {\it pack\_file\_name\/} and {\it pack\_buffered\_name\/}
(the functions that call {\it append\_to\_name\/}),
we have to ``unfreeze'' its value
if it was ``frozen''.
||if (k < 0) k = -k;||
\indent In {\it make\_name\_string},
each (multibyte) character from {\it name\_of\_file\/}
must be converted from \UTF-8 to Unicode,
before being converted to \TeX's internal encoding.
This is done using the C~library function
{\it mbtowc\/} [4].
\verbatim
< append_char(xord[name_of_file[k]]);

> { wchar_t wc;
>   k += mbtowc(&wc, name_of_file+k,
>               MB_CUR_MAX) - 1;
>   append_char(xord[wc]); }
\endverbatim
\medskip
In the code checking {\it format\_default\_length\/}
for consistency, we
use the C~library function {\it wcstombs\/} [4] to count
the number of bytes in the \UTF-8 representation of
{\it TEX\_format\_default\/}.
\verbatim
< if (format_default_length >
<     file_name_size)

> if (wcstombs(NULL,
>     TEX_format_default+1,0) >
>     file_name_size)
\endverbatim
\medskip
In the function {\it pack\_buffered\_name\/},
the code that drops excess characters
assumes that each character is one byte:
\verbatim
if (n+b-a+1+format_ext_length >
    file_name_size)
  b=a+file_name_size-n-1-format_ext_length;
\endverbatim
But the number of bytes used to represent
a character in \hbox{\UTF-8} may be more than one.
Therefore, we use an equivalent method to drop excess
characters, the one which will work with multibyte characters:
After appending
the contents of
{\it buffer\/$[a\mathrel{.\,.}b]$\/}
to {\it name\_of\_file\/}, we roll back in it
character by character until the
format extension fits in it. We use the C~library
function {\it mblen\/} [4] to determine
the start of the next (multibyte) character to be discarded.
\verbatim
while (k+wcstombs(NULL,TEX_format_default+
       format_default_length-
       format_ext_length+1,0) >
       file_name_size) {
  k--;
  while (mblen(name_of_file+k+1,MB_CUR_MAX)
         ==-1)
    k--;
}
\endverbatim

\head * References *

\begingroup
\vskip-1pt

\frenchspacing
\parindent=0pt
\parskip=4pt plus4pt
\def\\{\hfil\break}
\everypar{\hangindent=1.5em\relax}
\raggedright

[1] Ruckert, Martin. WEB to cweb.\\
{\tt mruckert.userweb.mwn.de/hint/web2w.html}

[2] Source of the present implementation.\\
{\tt https://github.com/igor-liferenko/tex}

[3] Knuth, Donald E. {\sl \TeX: The Program\/}, 1986. ISBN~0201134373.

[4] Single Unix Specification. Introduction to ISO~C Amendment 1
(Multibyte Support Environment).\\
{\tt\catcode`_=11 http://unix.org/version2/whatsnew/\\login_mse.html}

\par\endgroup

\advance\signaturewidth by 10pt
\makesignature
\endarticle
